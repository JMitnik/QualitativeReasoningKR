\section{Methodology}

\subsection{Assumptions and decisions}
Within the frame of this assignment, a number of assumptions were made when
implementing a set of rules for the generation of the state graph. The level of

Aside from the given constraints regarding the domain's relations, a number of
additional constraints were set as inferred by the existing ones. The exogenous
variable, the inflow, has been set to be a binary variable, which means that
either the sink is on in an instant, or it is off: this is to simplify the
unnecessary amount of states which may not add much to the model itself.
Furthermore, the exogenous variable will be modeled based off a negative
parabola. The reason a negative parabola is chosen, is twofold: a real
interaction with a sink would go in the way off washing hands (on), then turning
it off, and turning it on yet again after some time went by. As such, a negative
parabola starts this pattern immediately, and can represent the inactivity (drainage)
of a sink as well as the reactivation.

Furthermore, the representation to a real sink is taken further to the landmark
states of our container. If the container, and by proxy the outflow, reaches a
maximum or minimum, then the derivatives are instantly set within the same state
to zero. This way, the state will instantly achieve maximum without insinuating
that there is more to come, or less to come on a minimum state.

\subsection{Generating a state-graph}
The construction of a state-graph begins is initialized with an empty node. This
node contains a collection of the different magnitudes and derivatives of a
quantity, which is represented as a state. When the state-graph is started