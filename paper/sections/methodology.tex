\section{Methodology}

\subsection{Assumptions and decisions}
Within the frame of this assignment, a number of assumptions were made when
implementing a set of rules for the generation of the state graph. The level of

\textit{Domain assumptions}
Aside from the given constraints regarding the domain's relations, a number of
additional constraints were set as inferred by the existing ones. One such a
constraint is a quantity correspondence from the Container to the Outflow. Due
to the previously mentioned proportional
\textit{Intra-state assumptions}

% Assumption: Our exogenous variable is binary, either it is on or off.

% Assumption: Ambiguities will always be generated if an entity's derivative is
% already set and a relation tries to change the derivative Assumption.

% Assumption: An intra-state transition is that if an entity reaches its max and
% min landmark, its derivative instantly becomes 0.

\subsection{Generating a state-graph}