\section{Methodology}

\subsection{Assumptions and decisions}
Within the frame of this assignment, a number of assumptions were made when
implementing a set of rules for the generation of the state graph. The level of

Aside from the given constraints regarding the domain's relations, a number of
additional constraints were set as inferred by the existing ones. The exogenous
variable, the inflow, has been set to be a binary variable, which means that
either the sink is on in an instant, or it is off: this is to simplify the
unnecessary amount of states which may not add much to the model itself.
Furthermore, the exogenous variable will be modeled based off a negative
parabola. The reason a negative parabola is chosen, is twofold: a real
interaction with a sink would go in the way off washing hands (on), then turning
it off, and turning it on yet again after some time went by. As such, a negative
parabola starts this pattern immediately, and can represent the inactivity (drainage)
of a sink as well as the reactivation.

Furthermore, the representation to a real sink is taken further to the landmark
states of our container. If the container, and by proxy the outflow, reaches a
maximum or minimum, then the derivatives are instantly set within the same state
to zero. This way, the state will instantly achieve maximum without insinuating
that there is more to come, or less to come on a minimum state.

\subsection{Generating a state-graph}
\textit{Representing the state-graph}. A state-graph is initialized with an empty node. This node contains a collection
of the different magnitudes and derivatives of a quantity, which is referred to
as a state. By turning these values of the entities into a string, it is
possible to hash a state by the string representation of all these values. This
will be important when storing each state in a global variable `visitedStates',
which keeps track of new encountered variables. 

\textit{Generating new states}. Upon entering a state, the main goal is to apply
a number of functions on the value of the entities, and to generate as many
possible children as possible. This is done in three stages: first apply the
current derivatives on the children. If the magnitude value of an entity's
quantity is on a landmark (such as zero and maximum), this will produce a single
state. However, due to the nature of intervals as explained in
\cite{Bredeweg06garp3-}, the values could either transition to another value in
the interval or to the next state in the quantity space: as such, these possible
ambiguity leads to an additional child being generated to account for this
chance. The second stage is to apply the current relations to the states. This
can be intra-state and immediately (e.g. apply when no ambiguities are present),
or the examined state is split into multiple states with different outcomes.
First, any influence relations are applied, and then the 