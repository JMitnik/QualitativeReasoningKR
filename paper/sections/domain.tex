\section{The Domain: Causal Model}
\label{sec: domain}

\begin{figure}
    \includegraphics[width=65mm]{assets/causal_graph.png}
    \caption{Causal model of the sink}
    \label{fig:causal_model}
\end{figure}

The constraints as presented within the frame of this assignment consist of the
the following:
\begin{itemize}
    \item A P+ (positive proportional) relationship from Container to Outflow.
    This insinuates that whenever the Container's derivative, the outflow goes
    through a proportional change for its derivative.
    \item A I+ (influential positive) from Inflow to Container. Whenever the
    inflow is active, which is understood as having a magnitude greater than
    zero, the container will be influenced to increase (thus setting the
    derivative above zero).
    \item An I- relation from Outflow to Container. The existence of the outflow
    means that there water in the container is decreasing.
    \item A VC (value correspondence) from the containers max value to the outflow's
    max. If the container reaches the top, the outflow will also be at its strongest.
    \item A VC relation from the containers empty value to the outflow's empty.
    With no water in the container, there is no water to be flowing out.
\end{itemize}

Figure \ref{fig:causal_model} captures this relationship.