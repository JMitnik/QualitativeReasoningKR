\section{Results}
\subsection{Trace}
The trace in the algorithm can indicate a number of difference between two
states, such as quantities. The built-in traces either can focus on the state
itself, or between states, such as in figures \ref{fig:trace_intra.png}
\ref{fig:trace_inter.png} respectively.

\begin{figure}
    \includegraphics[width=65mm]{assets/trace_inter.png}
    \caption{Inter-state trace}
    \label{fig:trace_inter.png}
\end{figure}

\begin{figure}
    \includegraphics[width=65mm]{assets/trace_intra.png}
    \caption{Intra-state trace}
    \label{fig:trace_intra.png}
\end{figure}

\subsection{State graph}
\begin{figure*}
    \includegraphics[width=\textwidth,height=12cm]{assets/result.png}
    \caption{State-graph}
    \label{fig:state_graph.png}
\end{figure*}

The generated state graph resulted in approximately 37 states. Initially it
seems difficult to follow the state graph. From an implementation standpoint,
the intuition that accompanies the interpretation of the state graph is derived
from mostly building the model. By pruning the states, the state graph
represents the rules setup well and established.

It is interesting that not many clear termination or stable nodes can be found.
One of the more surprising findings aside from that is how many states are
generated from these three simple interactions between entities. The exogenous
variable's pattern (parabola) doesn't seem to have much impact on the state
graph rather than allowing for a larger number of extra states. 

There are often transitions between states in which they seem to fall back and
jump between each other. This often happens in intervals, where an interval
might refer often to itself, and just as easily could jump between landmark
states.

\section{Conclusion}
In this paper, we tried to recreate a state-graph. The state graph was based on
rules and constraints, and as such, the resulting state graph shows a promising 
although expected behaviour. The nodes showed no specific errors in particular, 
and from the report's understanding, seems to match how many nodes should be in 
this paper.